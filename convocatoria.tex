\documentclass[11pt,letterpaper]{report}
\usepackage[utf8x]{inputenc}
\usepackage[spanish]{babel}
\usepackage{latexsym,amsmath,amssymb,amsthm}
\usepackage{graphicx}
\usepackage{pifont}
\usepackage[pdftex=true,colorlinks=true,plainpages=false]{hyperref}
\hypersetup{urlcolor=blue}
\hypersetup{linkcolor=black}
\hypersetup{citecolor=black}
\usepackage{lastpage}
\usepackage{url}
\usepackage{anysize} 
\marginsize{2cm}{2cm}{0.5cm}{2.3cm}
\usepackage{fancyhdr}
\usepackage{anyfontsize}
\usepackage{tocbibind}
\usepackage{eso-pic}
\usepackage{mathptmx}
\usepackage{draftwatermark}
%\usepackage{lipsum}
%-------------------------
\pagestyle{fancy}
\lhead{}
\chead{{\tt \Huge \bf Universidad Mayor de San Simón}\\{\tt \huge Facultad de Ciencias y Tecnología}\\{\tt \Large Sociedad Científica de Estudiantes de Sistemas e Informática}}
\rhead{}
\lfoot{}
\cfoot{{\small \tt Prolongación calle Sucre - Parque la Torre - Teléfono: 70776989}\\{\tt Bloque central de la FCyT - tercer piso}\\{\large \tt \url{http://scesi.fcyt.umss.edu.bo}}}
\rfoot{\thepage/\pageref{LastPage}}
\renewcommand{\headrulewidth}{0.4pt}
\renewcommand{\footrulewidth}{0.9pt}
\footskip = 32pt
\headheight = 56pt
\headsep = 8pt
%--------------------------


\newcommand\BackgroundPic{
\put(485,727){
\parbox[b][\paperheight]{\paperwidth}{%
%\vfill
%\centering
\includegraphics[scale=0.034]{img/logo3.png}%
%\vfill
}}}

%---------------------------
\SetWatermarkAngle{45}
%\SetWatermarkLightness{0.9}
%\SetWatermarkFontSize{1cm}
\SetWatermarkScale{3}
\SetWatermarkText{\includegraphics[scale=0.05]{img/logo2.png}}
%--------------------------------------

\begin{document}
\AddToShipoutPicture{\BackgroundPic}
\begin{center}
{\bf \Large CONVOCATORIA A NUEVOS POSTULANTES A LA SOCIEDAD CIENTÍFICA DE LA CARRERA DE SISTEMAS E INFORMÁTICA
(SCESI)}\\
\end{center}

{\bf \small ANTECEDENTES}\\

La Sociedad de estudiantes de Sistemas e Informática (SCESI), está conformada por estudiantes de las carreras de Sistemas e Informática, creada con el fin de incentivar y promover la investigación científica en estas carreras y a la vez generar una cultura de investigación.\\

Durante el primer semestre 2012, La SCESI está trabajando sobre las siguientes áreas de interés:
\begin{itemize}
\item Programación Web.
\item Electrónica – Informática.
\item Video - Streaming.
\item Virtualización de Servidores.
\item Seguridad Informática.
\end{itemize}

{\bf \small REQUISITOS}
\begin{itemize}
\item[a.] Ser estudiante de la(s) Carrera(s) de Lic. Informática y/o Sistemas, que cursa regularmente la Universidad.
\item[b.] Demostrar o manifestar interés por la investigación.
\item[c.] Presentar una carta dirigida al Presidente de la Comisión de Admisión en la que se manifieste que desea ingresar a la Sociedad Científica de Estudiantes de Sistemas Informática como integrante.
\item[d.] Llenar el formulario de nuevos integrantes.
\item[e.] Presentarse a la entrevista con la Comisión de admisión en la fecha indicada en la convocatoria.
\end{itemize}

{\bf \small DOCUMENTOS REQUISITOS A PRESENTAR}
\begin{itemize}
\item[a.] Presentar Solicitud escrita especificando claramente la solicitud de ingreso así como el motivo que lo lleva a esa decisión a la que se postula:
\begin{itemize}
\item Carta dirigida a la Comisión de Admisión.
\item Todo interesado debe presentar una sola postulación, indicando el área de interés entre los que  trabaja la SCESI, o nombrando otra área de interés.
\end{itemize}
\item[b.] Presentar el Formulario de nuevos Integrantes, correctamente llenado.
\item[c.] Presentar fotocopia de la matrícula de estudiante regular del presente semestre.
\end{itemize}

{\bf \small FECHA Y LUGAR DE PRESENTACION DE DOCUMENTOS}
\begin{enumerate}
\item DE LA FORMA\\
Presentación de la documentación en sobre manila cerrado y rotulado con:
\begin{itemize}
\item[-] Nombre y apellidos completos, dirección, teléfono(s) y e-mail del postulante.
\item[-] El área de interés, con el cual es postulante se siente motivado.
\end{itemize}
\item DE LA FECHA Y LUGAR\\
La documentación deberá ser presentada hasta horas 17:00 del día 15 de Abril del 2012, en ambientes de SCESI (Bloque antiguo de FCyT, Último piso, subiendo las gradas lado derecho).
\end{enumerate}

{\bf \small ENTREVISTAS}\\

Las entrevistas se realizarán de acuerdo al cronograma:
\begin{itemize}
\item 24 de marzo del 2012
\item 31 de marzo del 2012
\item 7 de abril del 2012
\item 14 de abril del 2012
\end{itemize}
La asignación de fechas será de acuerdo a la fecha de entrega de documentos, y a criterio de la comisión de admisión.\\

{\bf \small SELECCIÓN}\\

Una vez concluido el trabajo, la comisión de Admisión decidirá qué estudiantes serán seleccionados para ingresar al grupo de trabajo considerando el grado de interés, compromiso y el área  de interés.\\

{\bf \small CRONOGRAMA GENERAL}

\begin{center}
\begin{tabular}{|l|l|}
\hline
{\bf Eventos} & {\bf Fechas}\\
\hline
Publicación de convocatoria. & 23 de Marzo del 2012\\
\hline
Presentación de documentos hasta: & 20 de Abril del 2012\\
\hline
Entrevistas. & 31 de Marzo del 2012 \\
\hline
& 7 de Abril del 2012\\
\hline
& 14 de Abril del 2012\\
\hline
& 21 de Abril del 2012\\
\hline
Resultados. & 23 de Abril del 2012\\
\hline
\end{tabular}
\end{center}
 \vspace{1.7cm}

\begin{minipage}{0.3\textwidth}
\begin{center}
-----------------------\\
Benjamín Pérez\\
Presidente Sociedad Científica\\
\end{center}
\end{minipage}
\begin{minipage}{0.9\textwidth}
\begin{center}
-----------------------\\
Gonzalo Nina M.\\
Director Académico\\
\end{center}
\end{minipage}
\vspace{1.7cm}
\begin{center}
-----------------------\\
Carlos Caballero\\
Comisión de Admisión\\
\end{center}



\end{document}
