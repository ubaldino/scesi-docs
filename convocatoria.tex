\documentclass[11pt,letterpaper]{article}
\usepackage[utf8x]{inputenc}
\usepackage[spanish]{babel}
\usepackage{latexsym,amsmath,amssymb,amsthm}
\usepackage{graphicx}
\usepackage{pifont}
\usepackage[pdftex=true,colorlinks=true,plainpages=false]{hyperref}
\hypersetup{urlcolor=blue}
\hypersetup{linkcolor=black}
\hypersetup{citecolor=black}
\usepackage{lastpage}
\usepackage{url}
\usepackage{anysize} 
\marginsize{2.5cm}{1.5cm}{1cm}{1.2cm}
\usepackage{fancyhdr}
\usepackage{anyfontsize}
\usepackage{tocbibind}
\usepackage{multicol}
\usepackage{wrapfig}
\usepackage[rflt]{floatflt}
\usepackage[abs]{overpic}
\usepackage{draftcopy}
\usepackage{eso-pic}
\usepackage{lipsum}

\pagestyle{fancy}
\lhead{\footnotesize \slshape \leftmark} \chead{} \rhead{\footnotesize \slshape \rightmark}
\lfoot{\url{http://scesi.fcyt.umss.edu.bo}} \cfoot{\includegraphics[scale=0.035]{img/scesi.png}} \rfoot{\thepage/\pageref{LastPage}}
\renewcommand{\headrulewidth}{0.4pt}
\renewcommand{\footrulewidth}{0.3pt}
\setlength{\headsep}{1\headsep}
\setlength{\footskip}{1\footskip}


\newcommand\BackgroundPic{
\put(0,0){
\parbox[b][\paperheight]{\paperwidth}{%
\vfill
\centering
%\includegraphics[width=\paperwidth,height=\paperheight,keepaspectratio]{img/scesi.png}%
\vfill
}}}
\usepackage{mathptmx}
\usepackage{draftwatermark}
 \usepackage[svgnames]{xcolor}
 \usepackage{fix-cm}
 \usepackage{lipsum}% para generar texto automáticamente

\usepackage{draftwatermark}
% Use the following to make modification
\SetWatermarkAngle{45}
\SetWatermarkLightness{0.9}
\SetWatermarkFontSize{1cm}
\SetWatermarkScale{3}
%\SetWatermarkText{BORRADOR}

\SetWatermarkText{\includegraphics[scale=0.07]{img/waterMark.png}}
%\SetWatermarkLightness{5}

%\title{Gnome}
%\author{Ubaldino Zurita}
%\date{\today}
%\draftcopyName{Marca de aguañoekjfñloasdfjdassdf}{30}
%\usebackgroundtemplate{includegraphics[width=paperwidth]{img/scesi.png}}
\begin{document}


\begin{center}
{\bf \Large CONVOCATORIA A NUEVOS POSTULANTES A LA SOCIEDAD CIENTÍFICA DE LA CARRERA DE SISTEMAS E INFORMÁTICA
(SCESI)}\\
\end{center}
~\\
{\bf \small ANTECEDENTES}\\

La Sociedad de estudiantes de Sistemas e Informática (SCESI), está conformada por estudiantes de las carreras de Sistemas e Informática, creada con el fin de incentivar y promover la investigación científica en estas carreras y a la vez generar una cultura de investigación.\\

Durante el primer semestre 2012, La SCESI está trabajando sobre las siguientes áreas de interés:
\begin{itemize}
\item Programación Web.
\item Electrónica – Informática.
\item Streaming.
\item Virtualización de Servidores.
\item Seguridad Informática.
\end{itemize}
~\\
{\bf \small REQUISITOS}
\begin{itemize}
\item[a.] Ser estudiante de la(s) Carrera(s) de Lic. Informática y/o Sistemas, que cursa regularmente la Universidad.
\item[b.] Demostrar o manifestar interés por la investigación científica.
\item[c.] Presentar una carta dirigida al Presidente de la Comisión de Admisión en la que se manifieste que desea ingresar a la Sociedad Científica de Estudiantes de Sistemas Informática como miembro.
\item[d.] Llenar el formulario de nuevos miembros.
\item[e.] Presentarse a la entrevista con la Comisión de admisión en la fecha indicada en la convocatoria.
\end{itemize}
~\\
{\bf \small DOCUMENTOS REQUISITOS A PRESENTAR}
\begin{itemize}
\item[a.] Presentar Solicitud escrita especificando claramente la solicitud de ingreso así como el motivo que lo lleva a esa decisión a la que se postula:
\begin{itemize}
\item Carta dirigida a la Comisión de Admisión.
\item Todo interesado debe presentar una sola postulación, indicando el área de interés entre los que  trabaja la SCESI, o nombrando otra área de interés.
\end{itemize}
\item[b.] Presentar el Formulario de nuevos Integrantes, correctamente llenado.
\item[c.] Presentar fotocopia de la matrícula de estudiante regular del presente semestre.
\end{itemize}
~\\
{\bf \small FECHA Y LUGAR DE PRESENTACION DE DOCUMENTOS}
\begin{enumerate}
\item DE LA FORMA
Presentación de la documentación en sobre manila cerrado y rotulado con:
\begin{itemize}
\item[-] Nombre y apellidos completos, dirección, teléfono(s) y e-mail del postulante.
\item[-] El área de interés, con el cual es postulante se siente motivado.
\end{itemize}
\item DE LA FECHA Y LUGAR
La documentación deberá ser presentada hasta horas 17:00 del día 15 de Abril del 2012, en ambientes de SCESI (Bloque antiguo de FCyT, Último piso, subiendo las gradas lado derecho).
\end{enumerate}
~\\
{\bf \small ENTREVISTAS}\\

Las entrevistas se realizarán de acuerdo al cronograma:
\begin{itemize}
\item 24 de marzo del 2012
\item 31 de marzo del 2012
\item 7 de abril del 2012
\item 14 de abril del 2012
\end{itemize}
La asignación de fechas será de acuerdo a la fecha de entrega de documentos, y a criterio de la comisión de admisión.\\
~\\
{\bf \small SELECCIÓN}\\

Una vez concluido La comisión de Admisión decidirá qué estudiantes serán seleccionados para ingresar al grupo de trabajo, con el área de interés seleccionado de la SCESI, considerando el grado de interés, compromiso.\\
~\\
{\bf \small CRONOGRAMA GENERAL}
\begin{center}
\begin{tabular}{|l|l|}
\hline
Eventos & Fechas\\
\hline
Publicación de convocatoria. & 16 de Marzo del 2012\\
\hline
Presentación de documentos hasta: & 13 de Abril del 2012\\
\hline
Entrevistas. & 24 de Marzo del 2012 \\
\hline
& 31 de Marzo del 2012\\
\hline
& 7 de Abril del 2012\\
\hline
& 14 de Abril del 2012\\
\hline
Resultados. & 16 de Abril del 2012\\
\hline
\end{tabular}
\end{center}
 \vspace{2cm}

\begin{minipage}{0.3\textwidth}
\begin{center}
-----------------------\\
Benjamín Pérez\\
Presidente Sociedad Científica\\
\end{center}
\end{minipage}
\begin{minipage}{0.9\textwidth}
\begin{center}
-----------------------\\
Gonzalo Nina M.\\
Director Académico\\
\end{center}
\end{minipage}
\vspace{2cm}
\begin{center}
-----------------------\\
Carlos Caballero\\
Comisión de Admisión\\
\end{center}
\end{document}