\documentclass[11pt,letterpaper]{article}
\usepackage[utf8]{inputenc}
\usepackage[spanish]{babel}
\usepackage{latexsym,amsmath,amssymb,amsthm}
\usepackage{graphicx}
\usepackage{pifont}
\usepackage[pdftex=true,colorlinks=true,plainpages=false]{hyperref}
\hypersetup{urlcolor=blue}
\hypersetup{linkcolor=black}
\hypersetup{citecolor=black}
\usepackage{lastpage}
\usepackage{url}
\usepackage{anysize}
\marginsize{2cm}{2cm}{0.5cm}{2.3cm}
\usepackage{fancyhdr}
\usepackage{anyfontsize}
\usepackage{tocbibind}
\usepackage{eso-pic}
\usepackage{mathptmx}
\usepackage{draftwatermark}
\usepackage{multirow}
%\usepackage[usenames,dvipsnames]{xcolor}
%\usepackage{lipsum}
%-------------------------
\setlength{\parindent}{0in}
\pagestyle{fancy}
\lhead{}
\chead{{\Huge \tt Universidad Mayor de San Simón}\\{\tt \huge Facultad de Ciencias y Tecnología}\\{\tt \Large Sociedad Científica de Estudiantes de Sistemas e Informática}}
\rhead{}
\lfoot{}
\cfoot{{\small \tt Prolongación calle Sucre - Parque la Torre - Teléfono: 70776989}\\{\small \tt Bloque central de la FCyT - tercer piso}\\{\Large \tt \url{http://www.scesi.org}}}
\rfoot{\thepage/\pageref{LastPage}}
\renewcommand{\headrule}{{\color{black}\hrule width\headwidth height\headrulewidth \vskip-\headrulewidth}}
\renewcommand{\footrule}{{\color{black}%
\vskip-\footruleskip\vskip-\footrulewidth
\hrule width\headwidth height\footrulewidth\vskip\footruleskip}}
\renewcommand{\headrulewidth}{3pt}
\renewcommand{\footrulewidth}{3pt}
\footskip = 32pt
\headheight = 56pt
\headsep = 8pt
%--------------------------


\newcommand\BackgroundPic{
\put(497,728){
\parbox[b][\paperheight]{\paperwidth}{%
%\vfill
%\centering
\includegraphics[scale=0.9]{img/1.png}%
%\vfill
}}}

%---------------------------
\SetWatermarkAngle{0}
%\SetWatermarkLightness{0.9}
%\SetWatermarkFontSize{1cm}
\SetWatermarkScale{3}
\SetWatermarkText{\includegraphics[scale=0.4]{img/2.png}}
%--------------------------------------

\begin{document}
\AddToShipoutPicture{\BackgroundPic}
%	\tableofcontents
~\\
~\\
\begin{center}
\section*{Presentación}
\end{center}
La SCESI (sociedad científica de estudiantes), es una sociedad que la componen estudiantes, que tienen como objetivo el realizar investigación en todo lo relacionado a las ciencias de la computación y ademas a la difusión del software libre. Estamos reconocidos oficialmente dentro la estructura universitaria y tenemos estrecha relación con: DICyT, Decanato, Dirección de carreras de Informática e Ingeniería de Sistemas y ademas llevamos una estrecha relación con el centro de investigación MEMI.\\

\section{Proyecto de capacitación GNU/linux (Ubuntu)}
Esta iniciativa surgio de una manera natural, con el solo hecho de  que se vio a varios profesores mencionar de que GNU/Linux es complicado de usar.
\subsection{Antecedentes}
La falta de conocimiento del uso de GNU/Linux en los profesores, hace que este sistema quede inservible para sus fines laborales, es mas algunos le tienen temor a GNU/Linux por que parece ser complicado en un principio. 
\subsection{Definición de proyecto}
El proyecto de capacitación esta elaborado para que cualquier profesor que quiera capacitación sobre el uso de GNU/Linux, pueda acudir a nosotros para su capacitación. El cual constara de varias etapas para poder hacer una buena transferencia de conocimiento en el uso del sistema operativo GNU/Linux.\\

\section{Proceso de elaboración el proyecto}
La capacitación consta mas que todo en el uso ágil del sistema operativo GNU/Linux.\\
Inicialmente proponemos el siguiente esquema de capacitación.
\subsection{Instalación} Este punto es importante por que sucede que a algunos se les ocurre reinstalar windows y dan por perdido a GNU/Linux.
\begin{itemize}
\item Sistemas de ficheros
\item Particionado de discos duros
\item Instalar GNU/Linux junto a otro sistema operativo
\item Gestor de arranque (datallado)
\item charla de sucesos que no se debe realizar a la hora de instalar GNU/Linux
\end{itemize}
\subsection{GNU/Linux sin interfaz gráfica}
\begin{itemize}
\item navegar por las ttys
\item uso de los comandos básicos
\item uso de comandos intermedios
\end{itemize}

\subsection{Uso de la interfaz gráfica en GNU/Linux}
\begin{itemize}
\item navegar de forma rápida y eficiente en la interfaz gráfica
\item vista general de las aplicaciones
\item búsqueda eficiente de las aplicaciones
\item navegar entre las carpetas
\item manejar los discos duros y sus particiones
\item personalizar la interfaz gráfica
\item administrar dispositivos integrados y periféricos
\item tips de los navegadores de internet
\item utilitarios de servicios en la nube
\end{itemize}
\subsection{Uso de los paquetes de ofimática}
Para los paquetes de ofimática se usará la suite de Libre Office.
\begin{itemize}
\item Writer
\item Calc
\item Draw
\item Impress
\item Math
\end{itemize}

\subsection{Edición de contenido multimedia}
\begin{itemize}
\item inkscape
\item gimp
\item transmagedon
\end{itemize}

\subsection{GNU/Linux y sus ventajas}
\begin{itemize}
\item uso de comandos avanzados
\item fusionar comandos
\item uso detallado de los servicios en linux
\item instalación de programas
\item manejo repositorios
\end{itemize}

\section{Cronograma de trabajo}
Con respecto al cronograma de trabajo, la SCESI consta con equipos de capacitación para la Olimpiada Boliviana de informática, esto es mas para estar a disposición en cualquier horario para llevar a cabo la capacitación.\\
Recalcando esta situación dejamos la elección del cronograma, a los profesores que deseen capacitarse.



\vspace{7cm}

\begin{minipage}{0.25\textwidth}
\begin{center}
------------------------------\\
{\bf Gonzalo Nina M.}\\
Presidente (scesi)\\
\end{center}
\end{minipage}
\begin{minipage}{0.47\textwidth}
\begin{center}
------------------------------\\
{\bf Carlos Caballero}\\
Jefe de proyecto
\end{center}
\end{minipage}
\begin{minipage}{0.28\textwidth}

\begin{center}
------------------------------\\
{\bf Gonzalo Nina M.}\\
Director Académico\\
\end{center}
\end{minipage}
\end{document}
