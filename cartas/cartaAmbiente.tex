

\documentclass[a4paper,12pt]{letter}

\usepackage[spanish]{babel}
\usepackage{url}
\usepackage{hyperref}

\begin{document}

\begin{letter}{Ing. Hern\'an Flores Garcia\\ Director Acad\'emico de la Facultad de Ciencias y Tecnolog\'ia - UMSS \\ Presente. -}

\begin{center}
	\opening{\textbf{REF.: \underline{Sociedad Cient\'ifica de Estudiantes de Sistemas e Inform\'atica}}}
\end{center}

Estimado Director:

A tiempo de saludarlo muy atententamente, me dirijo a Usted con la finalidad de informale acerca de las 
actividades que realizamos como Sociedad Cient\'ifica de Estudiantes de Sistemas Inform\'atica - SCESI.

La SCESI, no es un Centro de estudiantes, ni un club de amigos.

La SCESI, es una sociedad que la componen estudiantes, que tienen como objetivo el realizar investigaci\'on en
todo lo relacionado a las ciencias de la computaci\'on. Estamos reconocidos oficialmente dentro la estructura 
universitaria y tenemos estrecha relacion con: DICyT, Decanato, Direcci\'on de carreras de Inform\'atica e 
Ingenier\'ia de Sistemas.

La SCESI, es totalmente neutral en todo lo que es la pol\'itica de la UMSS, es decir, no estamos a favor ni en 
contra de ningun frente pol\'itico ya sean de Docentes o Estudiantes, y es totalmente abierta a todo estudiante 
que desea hacer investigaci\'on en Ciencias de la Computaci\'on.

Algunas areas de investigaci\'on que cubrimos son:

\begin{itemize}
    \item Programaci\'on
    \item Sistemas Operativos como ser Linux, este campo es amplio, como ser:
    \begin{itemize}
        \item Instalaci\'on de una distribuci\'on de Linux.
        \item Configurar codec's de audio y video.
        \item Instalacion de herramientas de programaci\'on.
        \item Compilar el kernel(nucleo de linux).
        \item etc.
    \end{itemize}
    \item Programaci\'on en el campo de la Electr\'onica
    \item etc.
\end{itemize}

Ademas de lo ya mencionado, la SCESI tiene como objetivo principal el hacer investigaci\'on en el area de 
Ciencias de la Computaci\'on, como ya lo demostramos el a\~no 2008 en la f\'eria de Tecnolog\'ia ganando el 
tercer lugar con el proyecto ``Octopus - Terminales tontas'' en el \'area Tecnolo\'ia de los Ordenadores. 

El pasado a\~no 2010 en el mes de Abril fuimos invitados cuatro miembros de la SCESI al {\bfseries Free Software 
Asunci\'on} en Paraguay.

La SCESI tambien a trabajado hombro a hombro junto con la ``Comunidad Haskell San Sim\'on'', esta comunindad
se destaca por el uso de programaci\'on funcional y el desarrollo de aplicaciones usando este paradigma y 
la investigaci\'on de lenguajes de programaci\'on.

Actualmente la SCESI esta ubicada en el ultimo piso del edificio facultativo, en la ex-aula 636.

{\bfseries Una de las principales necesidades actuales}, es poder contar con equipos de computacion y un ambiente seguro 
para la realizacion de las actividades que se llevan adelante.

Adjuntamos a la presente carta informes de nuestros proyectos.

\vspace{7.5cm}

\begin{center}
                \centerline{...................................}
                \centerline{Benjam\'in P\'erez}
                \centerline{{\bfseries Presidente de la SCESI}}
                \centerline{\url {http://scesi.fcyt.umss.edu.bo}}
\end{center}

\end{letter}

\end{document}
