\documentclass[a4paper,10pt]{letter}
\usepackage[spanish]{babel}
\usepackage[utf8x]{inputenc}
\usepackage[pdftex]{graphicx}
\usepackage{url}
\usepackage{hyperref}
\hypersetup{colorlinks,linkcolor=blue,pdfstartview=FitH,
                pdftitle=Software libre en la educacion,pdfauthor=Benjam\'in P\'erez,pdfsubject=Software Libre}

\begin{document}
%\date{18 de marzo de 2010}
\date{7 de septiembre de 2010}
% If you want headings on subsequent pages,
% remove the ``%'' on the next line:
% \pagestyle{headings}

\begin{letter}{Acad\'emicos: \\ Software Libre en la educaci\'on \\ \underline{Presente}.-}
%\address{$$ADDRESS$$}
%asdf
\begin{center}
\opening{\underline{{\bfseries REF.:  PARTICIPACI\'ON EN CONFERENCIAS}}}
\end{center}
%\opening{$$SALUTATION$$}

Mediante la presente les hago llegar un saludo cordial y éxito en sus funciones. Somos miembros de la Sociedad Científica de Estudiantes de Sistemas e Informática de la Universidad Mayor de San Simon, el motivo por lo cual solicito lo referente, es para concientizar a los académicos de los beneficios de usar el Software Libre e invitarle a ver/escuchar las conferencias (via streaming) desde su ordenador, estas conferencias estar\'an a cargo de académicos de diferentes universidades de otros países, la cual contará con expertos en Software Libre que demuestran que se puede utilizar en la enseñanza.

Free Software Foundation de la India, SPACE, Thiruvananthapuram, y el Instituto Nacional de Tecnología, Calicut, están organizando una Conferencia Nacional de Software Libre en la Educación en Kozhikode durante el 10 al 12 septiembre 2010. La conferencia contará con eminentes académicos de los institutos nacionales de la India explicando la importancia de utilizar el Software Libre en la educación. La conferencia también contará con expertos en Software Libre que demuestra que se puede utilizar en la enseñanza superior. La inauguración será en la noche del 10. Richard Stallman, fundador del proyecto GNU y la Free Software Foundation, pronunciará un discurso después de la sesión inaugural. Favor ver el sitio {\bfseries \url{http://79.136.118.252/fsinedu/}} para obtener más información.

Con este movimiento se pretende hacer un documento entre todos los participantes del streaming en el cual todos firmaran común acuerdo en apoyo en el uso del Software Libre en la educación y llevar a cabo estrategias que hagan de las propuestas generadas en la reunión, realidades en beneficio de nuestra Universidad.

Sin otro particular, y no dudando de su atenci\'on de lo citado, me despido atentamente augur\'andoles \'exito en sus funciones.\\

\vspace{1cm}

%\signature{Jos\'e Benjam\'in P\'erez Soto \\SIS: 200206288 \\CI: 5282843}

\signature{\\$$TITLE$$}
%\centerline {....................................................}
\includegraphics [height=1.7cm, width=3.0cm] {logo_scesi.png}\\
\small {{\bfseries Sociedad Cient\'ifica de Estudiantes de Sistemas e Inform\'atica - UMSS}\\
Tel\'efono: 70776989}

%\signature{\\$$TITLE$$}
%\centerline {....................................................}
%\small {{\bfseries Jos\'e Benjam\'in P\'erez Soto}\\
%Presidente de la Sociedad Cient\'ifica de Estudiantes de Sistemas e Inform\'atica - UMSS\\
%Tel\'efono: 70776989}

\vspace{0.7cm}
\centerline {Cochabamba - Bolivia}
%\centerline {Jos\'e Benjam\'in P\'erez Soto.}
%\centerline {C.I.:  5282843 - Cbba}
%\centerline {70776989 - 4248525}

%\closing{$$CLOSING$$}


%enclosure listing
%\encl{}

\end{letter}
\end{document}

